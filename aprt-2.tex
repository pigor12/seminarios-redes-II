\documentclass[bookmarks=false,aspectratio=169,9pt]{beamer}
% ------- Tema -------
\usetheme{LightConsole}
% ------- Pacotes -------
\usepackage{xcolor}
\usepackage{ragged2e}
\usepackage{hyperref}
\usepackage{graphicx}
\usepackage{listings}
\usepackage{circuitikz}
\usepackage[brazil]{babel}
\usepackage[backend=biber,style=alphabetic,sorting=ynt]{biblatex}

\addbibresource{refs.bib}

\lstset{
  showstringspaces=false,
  captionpos=b,
  breaklines=true,
  language=Python,
  numbers=left,
  numberstyle=\tiny,
  tabsize=4
}
% ------- Meta -------
\date{10 de março, 2024}
\title[Seminário - Redes II]{Principais Ataques à Segurança da Informação na Atualidade}
\institute[Pontifícia Universidade Católica de Minas Gerais]{@pucminas}
\author{j.martins\footnote{João Victor Martins Medeiros}-pigor\footnote{Pedro Igor Martins dos Reis}}
\subtitle{Redes II}
\hypersetup{
    bookmarks=true,
    colorlinks=true,
    pdfpagemode=FullScreen,
    pdfsubject={Seminário - Redes II},
}
% ------- Documento -------
\begin{document}
\begin{frame}[fragile]
	\begin{verbatim}
                                                        ^^
                    ^^      ..                                       ..
                            []                                       []
                          .:[]:_          ^^                       ,:[]:.
                        .: :[]: :-.                             ,-: :[]: :.
                      .: : :[]: : :`._                       ,.': : :[]: : :.
                    .: : : :[]: : : : :-._               _,-: : : : :[]: : : :.
                _..: : : : :[]: : : : : : :-._________.-: : : : : : :[]: : : : :-._
                _:_:_:_:_:_:[]:_:_:_:_:_:_:_:_:_:_:_:_:_:_:_:_:_:_:_:[]:_:_:_:_:_:_
                !!!!!!!!!!!![]!!!!!!!!!!!!!!!!!!!!!!!!!!!!!!!!!!!!!!![]!!!!!!!!!!!!!
                ^^^^^^^^^^^^[]^^^^^^^^^^^^^^^^^^^^^^^^^^^^^^^^^^^^^^^[]^^^^^^^^^^^^^
                            []                                       []
                            []                                       []
                            []                                       []
                 ~~^-~^_~^~/  \~^-~^~_~^-~_^~-^~_^~~-^~_~^~-~_~-^~_^/  \~^-~_~^-~~-
                 ~ _~~- ~^-^~-^~~- ^~_^-^~~_ -~^_ -~_-~~^- _~~_~-^_ ~^-^~~-_^-~ ~^
\end{verbatim}
\end{frame}
\begin{frame}
	\titlepage
\end{frame}
\section{Caso \textit{Western Digital}}
\subsection{Vítima}
\begin{frame}[fragile]{Western Digital}
	\begin{columns}
		\begin{column}{0.5\textwidth}
			\begin{verbatim}
                              ########
                            ####    ########
                    ############    ##########
                ############################
              ############################
          ##############################
        ################################
        ##########        ##############
      ############          ##########
      ##################    ##########
      ##################    ########
        ################    ######
          ############          ####
        ##########              ####
        ####################      ######
\end{verbatim}
		\end{column}
	\end{columns}
\end{frame}
\begin{frame}{Western Digital\footnote{\href{https://www.westerndigital.com/}{https://www.westerndigital.com/}}}
	\begin{columns}
		\begin{column}{0.5\textwidth}
			\begin{itemize}
				\item Fundada em 1970;
				\item Atuação em nível global;
				\item 20B USD em valor de mercado;
				\item Fabricante de diversas soluções de armazenamento;
				\item Grupo formado por mais 4 subsidiárias:
				      \begin{itemize}
					      \item WD;
					      \item WD\_Black;
					      \item SanDisk;
					      \item SanDisk Professional.
				      \end{itemize}
			\end{itemize}
		\end{column}
	\end{columns}
\end{frame}
\subsection{Atacante}
\begin{frame}[fragile]{Alphv, a.k.a BlackCat}
	\begin{columns}
		\begin{column}{0.5\textwidth}
			\begin{verbatim}
                  ##
    ##            ####
      ####          ####
        ########        ####
              ########  ######
                  ################
                  ##################
                  ####################
              ####      ######    ####
              ######    ######      ##
            ############################
          ######      ########    ########  ##
          ####################        ######
          ##########  ##########    ##########
        ########      ############      ######
        ##########################          ##
          ##############      ####
          ##############        ######
            ##########        ########
\end{verbatim}
		\end{column}
	\end{columns}
\end{frame}
\subsection{Ataque}
\begin{frame}{Dia do ataque}
	\begin{columns}
		\begin{column}{0.5\textwidth}
			\begin{itemize}
				\item 26 de março de 2023;
				\item 10TB de dados internos e SAP \textit{Backoffice} roubados\cite{bleepingcomputer_1};
				\item Serviços em nuvem\footnote{My Cloud, My Cloud Home, My Cloud OS 5, \dots} temporariamente desligados;
				\item \textit{"We know you have the link to our onion site. Approach with payment prepared, or **** off. Brace yourselves for the gradual fallout."}\cite{twitter};
				\item \textit{"We will **** you until you cannot stand anymore Western Digital. Consider this our final warning."}\cite{techcrunch};
			\end{itemize}
		\end{column}
	\end{columns}
\end{frame}
\begin{frame}{Medidas preventivas}
	\begin{columns}
		\begin{column}{0.5\textwidth}
			\begin{itemize}
				\item Treinamento de funcionários;
				\item Auditorias internas regulares;
				\item Estratégias de resposta a incidentes;
				\item Fortalecimento de sua infraestrutura interna.
			\end{itemize}
		\end{column}
	\end{columns}
\end{frame}
\subsection{Desafio}
\begin{frame}{Enunciado}
	Considere o ataque cibernético sofrido pela Western Digital em 2023, qual das seguintes afirmações é \textbf{incorreta}?
	\begin{itemize}
		\item \textbf{A)} A Western Digital utilizou o incidente como uma oportunidade para melhorar suas práticas de segurança cibernética, implementando criptografia de arquivos em toda a sua rede como medida de proteção contra futuros ataques.
		\item \textbf{B)} O grupo de \textit{ransomware} 'Alphv', também conhecido como 'BlackCat', assumiu a responsabilidade pelo ataque, utilizando o incidente para extorquir a empresa, ameaçando publicar os dados roubados.
		\item \textbf{C)} O ataque resultou na exfiltração de aproximadamente 10 terabytes de dados da empresa, incluindo informações detalhadas de clientes, e os hackers exigiram um resgate de oito dígitos.
		\item \textbf{D)} A Western Digital desativou temporariamente alguns de seus serviços para conter a ameaça e iniciou uma investigação com especialistas em cibersegurança e autoridades policiais.
	\end{itemize}
\end{frame}
\begin{frame}{Resposta}
	Considere o ataque cibernético sofrido pela Western Digital em 2023, qual das seguintes afirmações é \textbf{incorreta}?
	\begin{itemize}
		\item \textbf{A)} A Western Digital utilizou o incidente como uma oportunidade para melhorar suas práticas de segurança cibernética, implementando criptografia de arquivos em toda a sua rede como medida de proteção contra futuros ataques.
		\item \textbf{B)} O grupo de \textit{ransomware} 'Alphv', também conhecido como 'BlackCat', assumiu a responsabilidade pelo ataque, utilizando o incidente para extorquir a empresa, ameaçando publicar os dados roubados.
		\item \textcolor{green}{\textbf{C)} O ataque resultou na exfiltração de aproximadamente 10 terabytes de dados da empresa, incluindo informações detalhadas de clientes, e os hackers exigiram um resgate de oito dígitos.}
		\item \textbf{D)} A Western Digital desativou temporariamente alguns de seus serviços para conter a ameaça e iniciou uma investigação com especialistas em cibersegurança e autoridades policiais.
	\end{itemize}
\end{frame}
\subsection{Bibliografia}
\begin{frame}{Referências}
	\printbibliography
\end{frame}
\end{document}

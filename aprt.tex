\documentclass[bookmarks=false,aspectratio=169,9pt]{beamer}
% ------- Tema -------
\usetheme{DarkConsole}
% ------- Pacotes -------
\usepackage{xcolor}
\usepackage{ragged2e}
\usepackage{hyperref}
\usepackage{graphicx}
\usepackage{listings}
\usepackage{circuitikz}
\usepackage[brazil]{babel}
\usepackage[backend=biber,style=alphabetic,sorting=ynt]{biblatex}

\addbibresource{refs.bib}

\lstset{
  showstringspaces=false,
  captionpos=b,
  breaklines=true,
  language=Python,
  numbers=left,
  numberstyle=\tiny,
  tabsize=4
}
% ------- Meta -------
\date{9 de março, 2024}
\title[Seminário - Redes II]{Principais Ataques à Segurança da Informação na Atualidade}
\institute[Pontifícia Universidade Católica de Minas Gerais]{@pucminas}
\author{j.martins\footnote{João Victor Martins Medeiros}-pigor\footnote{Pedro Igor Martins dos Reis}}
\subtitle{Redes II}
\hypersetup{
    bookmarks=true,
    colorlinks=true,
    pdfpagemode=FullScreen,
    pdfsubject={Seminário - Redes II},
}
% ------- Documento -------
\begin{document}
\begin{frame}[fragile]
	\begin{columns}
		\begin{column}{\textwidth}
			\begin{verbatim}
                                              |&gt;&gt;&gt;
                                              |
                                |&gt;&gt;&gt;      _  _|_  _         |&gt;&gt;&gt;
                                |        |;| |;| |;|        |
                            _  _|_  _    \\.    .  /    _  _|_  _
                          |;|_|;|_|;|    \\:. ,  /    |;|_|;|_|;|
                          \\..      /    ||;   . |    \\.    .  /
                            \\.  ,  /     ||:  .  |     \\:  .  /
                            ||:   |_   _ ||_ . _ | _   _||:   |
                            ||:  .|||_|;|_|;|_|;|_|;|_|;||:.  |
                            ||:   ||.    .     .      . ||:  .|
                            ||: . || .     . .   .  ,   ||:   |       \,/
                            ||:   ||:  ,  _______   .   ||: , |            /`\
                            ||:   || .   /+++++++\    . ||:   |
                            ||:   ||.    |+++++++| .    ||: . |
                          __ ||: . ||: ,  |+++++++|.  . _||_   |
                ____--`~    '--~~__|.    |+++++__|----~    ~`---,
            -~--~                   ~---__|,--~'                  ~~----_____-~'
\end{verbatim}
		\end{column}
	\end{columns}
\end{frame}
\begin{frame}
	\titlepage
\end{frame}
\section{Caso \textit{Insomniac Games}}
\subsection{Vítima}
\begin{frame}[fragile]{Insomniac Games, Inc.}
	\begin{columns}
		\begin{column}{0.5\textwidth}
			\begin{verbatim}
                            ### ##
                            #######
               ########### #########
             #########################
           #############################
          ###############################
         #################################
         ##################################
         ##########################
         #########################
   ##############################
 ##################################
###        ###############   #######
#############      #########    ##
 #################     #####
    ################
                 #####
\end{verbatim}
		\end{column}
	\end{columns}
\end{frame}
\begin{frame}{Insomniac Games, Inc.\footnote{\href{https://insomniac.games/}{https://insomniac.games/}}}
	\begin{columns}
		\begin{column}{0.5\textwidth}
			\begin{itemize}
				\item Fundada em 1994;
				\item 400+ funcionários;
				\item PlayStation Studios;
				\item Xtreme Software, Inc.;
				\item Desenvolvedora de jogos eletrônicos, como:
				      \begin{itemize}
					      \item \textit{Sunset Overdrive} (2014);
					      \item \textit{Ratchet \& Clank} (2002-);
					      \item \textit{Marvel's Spider-Man} (2018-);
					      \item \textit{Spyro The Dragon} (1996-2000);
					      \item \dots
				      \end{itemize}
			\end{itemize}
		\end{column}
	\end{columns}
\end{frame}
\subsection{Atacante}
\begin{frame}[fragile]{Rhysida}
	\begin{columns}
		\begin{column}{0.5\textwidth}
			\begin{verbatim}
        ############################
      ###$######         ############
      ##$#$####              #########
     ##$##$###                ########
    ##$####$###               ########
   ##$####$####               ########
   ###$##$####               #########
   Ø###$###Ø#               #########
   ###$#$###               #########
    #######               #########
   #######               #########
    ####               ##########
     $               ###########
    $              ###########
    $            ###########
   $           ###########
  $ $        ###########
           ###########
\end{verbatim}
		\end{column}
	\end{columns}
\end{frame}
\begin{frame}{Rhysida}
	\begin{columns}
		\begin{column}{0.5\textwidth}
			\begin{itemize}
				\item Bitcoin;
				\item \textit{"Gold Victor"};
				\item \textit{"Vice Society"};
				\item \textit{Ransomware} com extorção dupla;
				\item (Provável) Origem Russa, Bielorussa e Cazaquistanês;
				\item De acordo com a CISA\footnote{Cybersecurity \& Infrastructure Security Agency}, surgiu em 2021\cite{cisa_advisory};
			\end{itemize}
		\end{column}
	\end{columns}
\end{frame}
\begin{frame}{Modus operandi}
	\begin{figure}[!ht]
		\centering
		\resizebox{0.8\textwidth}{!}{%
			\begin{circuitikz}
				\tikzstyle{every node}=[font=\LARGE]
				\node [] at (-6,12.5) {Contato};
				\node [] at (-6,14.5) {\textit{Phishing}};
				\node [] at (2,12.5) {Movimento Lateral};
				\node [] at (2,14.5) {\textit{Cobalt Strike}};
				\node [] at (2,10.5) {PsExec};
				\node [] at (12,14.5) {\textit{Ransomware}};
				\node [] at (12,10.5) {Evasão};
				\node [] at (12,16.5) {Chave 4096-Bit\footnote{Criptografia RSA \& AES-CTR}};
				\node [] at (12,8.5) {\textit{Scripts}};
				\draw [->,  >=Stealth] (-4,12.5) -- (-1.25,12.5);
				\draw [<->, >=Stealth] (12,13.75) -- (12,11.25);
				\draw [](12,12.5)  to[short] (5,12.5);
				\draw [](-6,13.25) to[short, -o] (-6,14);
				\draw [](2,13.25)  to[short, -o] (2,14);
				\draw [](2,11.25)  to[short, o-] (2,12);
				\draw [](12,15.25) to[short, -o] (12,15.75);
				\draw [](12,9.25)  to[short, o-] (12,9.75);
			\end{circuitikz}
		}
	\end{figure}
\end{frame}
\begin{frame}[fragile]{Modus operandi}
	\begin{center}
		\begin{tabular}{c|p{9cm}}
			\verb|conhost.exe|\footnote{O conhost é um arquivo executável legítimo do Windows}             & Um binário de \textit{ransomware}.                         \\
			\verb|psexec.exe|\footnote{Psexec também é uma ferramenta originalmente legítima da Microsoft} & Executa um processo local ou remotamente.                  \\
			\verb|S_0.bat|                                                                                 & Um \textit{script} para preparação de \textit{ransomware}. \\
			\verb|1.ps1|                                                                                   & Identifica os arquivos a serem criptografados.             \\
			\verb|S_1.bat|                                                                                 & Copia \verb|conhost.exe| para \verb|C:\Windows\Temp|.      \\
			\verb|S_2.bat|                                                                                 & Adiciona a extensão \verb|.Rhysida| em todo o ambiente.
		\end{tabular}
	\end{center}
\end{frame}
\begin{frame}[fragile]{Modus operandi}
	\begin{center}
		\begin{lstlisting}[caption=Regras YARA fornecidas pela equipe do SentinelOne]
        rule rw_rhysida {
            meta:
              author      = "Alex Delamotte"
              description = "Rhysida ransomware detection."
              sample      = "69b3d913a3967153d1e91ba1a31ebed839b297ed"
              reference   = "https://s1.ai/rhys"
            strings:
              $typo1 = { 63 6D 64 2E 65 78 65 20 2F ... }
              $cmd1  = { 63 6D 64 2E 65 78 65 20 2F ... }
              $cmd2  = { 63 6D 64 2E 65 78 65 20 2F ... }
              $byte1 = { 48 8D 05 72 AA 05 00 48 8B ... }
              $byte2 = { 48 8D 15 89 CF 03 00 48 89 ... }
            condition:
                2 of them
        }
\end{lstlisting}
	\end{center}
\end{frame}
\subsection{Ataque}
\begin{frame}{Dia do ataque}
	\begin{columns}
		\begin{column}{0.5\textwidth}
			\begin{itemize}
				\item \textit{X-Men} (2030)\cite{ign_insomniac};
				\item 50 Bitcoins\footnote{09/03, $\approx R\$17.296.275,91$};
				\item 7 dias para reagir;
				\item 1.6 TB de dados roubados\cite{engadget_insomniac};
				\item \textit{Marvel’s Wolverine} (2026);
				\item \textit{Venom: Lethal Protector} (2025)\cite{};
				\item Planejamento de uma década liberado;
				\item Confirmado no dia 12 de dezembro de 2023;
				\item 98\% dos dados recolhidos foram publicados;
				\item Passaportes e documentos governamentais de funcionários\cite{polygon_insomniac};
			\end{itemize}
		\end{column}
	\end{columns}
\end{frame}
\begin{frame}{Medidas preventivas}
	\begin{columns}
		\begin{column}{0.5\textwidth}
			\begin{itemize}
				\item Política \textit{Zero-Trust};
				\item Autenticação de multifator;
				\item Sistemas de \textit{anti-phishing}\cite{csoonline_antiphishing}\footnote{Avanan, Barracuda Sentinel, Cofense PDR, \dots};
				\item Treinamento de funcionários;
			\end{itemize}
		\end{column}
	\end{columns}
\end{frame}
\begin{frame}{Enunciado}
	Qual das seguintes afirmações abaixo é verdadeira?
	\begin{itemize}

		\item \textbf{A)} Grupos hackers, incluindo \textit{Rhysida}, empregam técnicas de phishing como seu principal meio de ganhar acesso a sistemas de informação, visando especificamente indivíduos através de e-mails falsos que parecem ser de fontes legítimas.

		\item \textbf{B)} O \textit{ransomware} é um tipo de \textit{malware} que, uma vez instalado em um sistema de computador, criptografa arquivos do usuário e exige um pagamento em criptomoeda para a descriptografia, mas não permite aos hackers acesso aos sistemas afetados ou dados.

		\item \textbf{C)} Grupos como \textit{Rhysida} operam exclusivamente por motivos financeiros, evitando alvos que possam ter significativas repercussões sociais ou políticas para minimizar a atenção das autoridades de aplicação da lei.

		\item \textbf{D)} As campanhas de \textit{ransomware}, como as conduzidas por grupos hackers incluindo Rhysida, são notáveis por evitar o uso de técnicas de engenharia social, preferindo explorar vulnerabilidades de software sem interação direta com as vítimas.
	\end{itemize}
\end{frame}
\begin{frame}{Resposta Correta}
	Qual das seguintes afirmações abaixo é verdadeira?
	\begin{itemize}

		\item \textcolor{green}{\textbf{A)} Grupos hackers, incluindo \textit{Rhysida}, empregam técnicas de phishing como seu principal meio de ganhar acesso a sistemas de informação, visando especificamente indivíduos através de e-mails falsos que parecem ser de fontes legítimas.}

		\item \textbf{B)} O \textit{ransomware} é um tipo de \textit{malware} que, uma vez instalado em um sistema de computador, criptografa arquivos do usuário e exige um pagamento em criptomoeda para a descriptografia, mas não permite aos hackers acesso aos sistemas afetados ou dados.

		\item \textbf{C)} Grupos como \textit{Rhysida} operam exclusivamente por motivos financeiros, evitando alvos que possam ter significativas repercussões sociais ou políticas para minimizar a atenção das autoridades de aplicação da lei.

		\item \textbf{D)} As campanhas de \textit{ransomware}, como as conduzidas por grupos hackers incluindo Rhysida, são notáveis por evitar o uso de técnicas de engenharia social, preferindo explorar vulnerabilidades de software sem interação direta com as vítimas.
	\end{itemize}
\end{frame}
\begin{frame}{Referências}
	\printbibliography
\end{frame}
\end{document}
